% Generated by Sphinx.
\def\sphinxdocclass{report}
\documentclass[letterpaper,10pt,english]{sphinxmanual}

\usepackage[utf8]{inputenc}
\ifdefined\DeclareUnicodeCharacter
  \DeclareUnicodeCharacter{00A0}{\nobreakspace}
\else\fi
\usepackage{cmap}
\usepackage[T1]{fontenc}
\usepackage{amsmath,amssymb}
\usepackage{babel}
\usepackage{times}
\usepackage[Bjarne]{fncychap}
\usepackage{longtable}
\usepackage{sphinx}
\usepackage{multirow}
\usepackage{eqparbox}


\addto\captionsenglish{\renewcommand{\figurename}{Fig. }}
\addto\captionsenglish{\renewcommand{\tablename}{Table }}
\SetupFloatingEnvironment{literal-block}{name=Listing }

\addto\extrasenglish{\def\pageautorefname{page}}

\setcounter{tocdepth}{1}


\title{Tracemac Documentation}
\date{April 26, 2016}
\release{2.0.0}
\author{Mattias Måhl}
\newcommand{\sphinxlogo}{}
\renewcommand{\releasename}{Release}
\makeindex

\makeatletter
\def\PYG@reset{\let\PYG@it=\relax \let\PYG@bf=\relax%
    \let\PYG@ul=\relax \let\PYG@tc=\relax%
    \let\PYG@bc=\relax \let\PYG@ff=\relax}
\def\PYG@tok#1{\csname PYG@tok@#1\endcsname}
\def\PYG@toks#1+{\ifx\relax#1\empty\else%
    \PYG@tok{#1}\expandafter\PYG@toks\fi}
\def\PYG@do#1{\PYG@bc{\PYG@tc{\PYG@ul{%
    \PYG@it{\PYG@bf{\PYG@ff{#1}}}}}}}
\def\PYG#1#2{\PYG@reset\PYG@toks#1+\relax+\PYG@do{#2}}

\expandafter\def\csname PYG@tok@mi\endcsname{\def\PYG@tc##1{\textcolor[rgb]{0.13,0.50,0.31}{##1}}}
\expandafter\def\csname PYG@tok@s2\endcsname{\def\PYG@tc##1{\textcolor[rgb]{0.25,0.44,0.63}{##1}}}
\expandafter\def\csname PYG@tok@cm\endcsname{\let\PYG@it=\textit\def\PYG@tc##1{\textcolor[rgb]{0.25,0.50,0.56}{##1}}}
\expandafter\def\csname PYG@tok@sx\endcsname{\def\PYG@tc##1{\textcolor[rgb]{0.78,0.36,0.04}{##1}}}
\expandafter\def\csname PYG@tok@vg\endcsname{\def\PYG@tc##1{\textcolor[rgb]{0.73,0.38,0.84}{##1}}}
\expandafter\def\csname PYG@tok@kp\endcsname{\def\PYG@tc##1{\textcolor[rgb]{0.00,0.44,0.13}{##1}}}
\expandafter\def\csname PYG@tok@m\endcsname{\def\PYG@tc##1{\textcolor[rgb]{0.13,0.50,0.31}{##1}}}
\expandafter\def\csname PYG@tok@sc\endcsname{\def\PYG@tc##1{\textcolor[rgb]{0.25,0.44,0.63}{##1}}}
\expandafter\def\csname PYG@tok@ge\endcsname{\let\PYG@it=\textit}
\expandafter\def\csname PYG@tok@nf\endcsname{\def\PYG@tc##1{\textcolor[rgb]{0.02,0.16,0.49}{##1}}}
\expandafter\def\csname PYG@tok@il\endcsname{\def\PYG@tc##1{\textcolor[rgb]{0.13,0.50,0.31}{##1}}}
\expandafter\def\csname PYG@tok@ow\endcsname{\let\PYG@bf=\textbf\def\PYG@tc##1{\textcolor[rgb]{0.00,0.44,0.13}{##1}}}
\expandafter\def\csname PYG@tok@nv\endcsname{\def\PYG@tc##1{\textcolor[rgb]{0.73,0.38,0.84}{##1}}}
\expandafter\def\csname PYG@tok@kr\endcsname{\let\PYG@bf=\textbf\def\PYG@tc##1{\textcolor[rgb]{0.00,0.44,0.13}{##1}}}
\expandafter\def\csname PYG@tok@gh\endcsname{\let\PYG@bf=\textbf\def\PYG@tc##1{\textcolor[rgb]{0.00,0.00,0.50}{##1}}}
\expandafter\def\csname PYG@tok@cpf\endcsname{\let\PYG@it=\textit\def\PYG@tc##1{\textcolor[rgb]{0.25,0.50,0.56}{##1}}}
\expandafter\def\csname PYG@tok@w\endcsname{\def\PYG@tc##1{\textcolor[rgb]{0.73,0.73,0.73}{##1}}}
\expandafter\def\csname PYG@tok@gd\endcsname{\def\PYG@tc##1{\textcolor[rgb]{0.63,0.00,0.00}{##1}}}
\expandafter\def\csname PYG@tok@mb\endcsname{\def\PYG@tc##1{\textcolor[rgb]{0.13,0.50,0.31}{##1}}}
\expandafter\def\csname PYG@tok@na\endcsname{\def\PYG@tc##1{\textcolor[rgb]{0.25,0.44,0.63}{##1}}}
\expandafter\def\csname PYG@tok@sr\endcsname{\def\PYG@tc##1{\textcolor[rgb]{0.14,0.33,0.53}{##1}}}
\expandafter\def\csname PYG@tok@err\endcsname{\def\PYG@bc##1{\setlength{\fboxsep}{0pt}\fcolorbox[rgb]{1.00,0.00,0.00}{1,1,1}{\strut ##1}}}
\expandafter\def\csname PYG@tok@gt\endcsname{\def\PYG@tc##1{\textcolor[rgb]{0.00,0.27,0.87}{##1}}}
\expandafter\def\csname PYG@tok@ss\endcsname{\def\PYG@tc##1{\textcolor[rgb]{0.32,0.47,0.09}{##1}}}
\expandafter\def\csname PYG@tok@ni\endcsname{\let\PYG@bf=\textbf\def\PYG@tc##1{\textcolor[rgb]{0.84,0.33,0.22}{##1}}}
\expandafter\def\csname PYG@tok@cp\endcsname{\def\PYG@tc##1{\textcolor[rgb]{0.00,0.44,0.13}{##1}}}
\expandafter\def\csname PYG@tok@gi\endcsname{\def\PYG@tc##1{\textcolor[rgb]{0.00,0.63,0.00}{##1}}}
\expandafter\def\csname PYG@tok@nn\endcsname{\let\PYG@bf=\textbf\def\PYG@tc##1{\textcolor[rgb]{0.05,0.52,0.71}{##1}}}
\expandafter\def\csname PYG@tok@s1\endcsname{\def\PYG@tc##1{\textcolor[rgb]{0.25,0.44,0.63}{##1}}}
\expandafter\def\csname PYG@tok@se\endcsname{\let\PYG@bf=\textbf\def\PYG@tc##1{\textcolor[rgb]{0.25,0.44,0.63}{##1}}}
\expandafter\def\csname PYG@tok@kt\endcsname{\def\PYG@tc##1{\textcolor[rgb]{0.56,0.13,0.00}{##1}}}
\expandafter\def\csname PYG@tok@s\endcsname{\def\PYG@tc##1{\textcolor[rgb]{0.25,0.44,0.63}{##1}}}
\expandafter\def\csname PYG@tok@gr\endcsname{\def\PYG@tc##1{\textcolor[rgb]{1.00,0.00,0.00}{##1}}}
\expandafter\def\csname PYG@tok@nc\endcsname{\let\PYG@bf=\textbf\def\PYG@tc##1{\textcolor[rgb]{0.05,0.52,0.71}{##1}}}
\expandafter\def\csname PYG@tok@nl\endcsname{\let\PYG@bf=\textbf\def\PYG@tc##1{\textcolor[rgb]{0.00,0.13,0.44}{##1}}}
\expandafter\def\csname PYG@tok@sd\endcsname{\let\PYG@it=\textit\def\PYG@tc##1{\textcolor[rgb]{0.25,0.44,0.63}{##1}}}
\expandafter\def\csname PYG@tok@gp\endcsname{\let\PYG@bf=\textbf\def\PYG@tc##1{\textcolor[rgb]{0.78,0.36,0.04}{##1}}}
\expandafter\def\csname PYG@tok@o\endcsname{\def\PYG@tc##1{\textcolor[rgb]{0.40,0.40,0.40}{##1}}}
\expandafter\def\csname PYG@tok@si\endcsname{\let\PYG@it=\textit\def\PYG@tc##1{\textcolor[rgb]{0.44,0.63,0.82}{##1}}}
\expandafter\def\csname PYG@tok@gu\endcsname{\let\PYG@bf=\textbf\def\PYG@tc##1{\textcolor[rgb]{0.50,0.00,0.50}{##1}}}
\expandafter\def\csname PYG@tok@nd\endcsname{\let\PYG@bf=\textbf\def\PYG@tc##1{\textcolor[rgb]{0.33,0.33,0.33}{##1}}}
\expandafter\def\csname PYG@tok@cs\endcsname{\def\PYG@tc##1{\textcolor[rgb]{0.25,0.50,0.56}{##1}}\def\PYG@bc##1{\setlength{\fboxsep}{0pt}\colorbox[rgb]{1.00,0.94,0.94}{\strut ##1}}}
\expandafter\def\csname PYG@tok@nb\endcsname{\def\PYG@tc##1{\textcolor[rgb]{0.00,0.44,0.13}{##1}}}
\expandafter\def\csname PYG@tok@ch\endcsname{\let\PYG@it=\textit\def\PYG@tc##1{\textcolor[rgb]{0.25,0.50,0.56}{##1}}}
\expandafter\def\csname PYG@tok@go\endcsname{\def\PYG@tc##1{\textcolor[rgb]{0.20,0.20,0.20}{##1}}}
\expandafter\def\csname PYG@tok@k\endcsname{\let\PYG@bf=\textbf\def\PYG@tc##1{\textcolor[rgb]{0.00,0.44,0.13}{##1}}}
\expandafter\def\csname PYG@tok@vc\endcsname{\def\PYG@tc##1{\textcolor[rgb]{0.73,0.38,0.84}{##1}}}
\expandafter\def\csname PYG@tok@c1\endcsname{\let\PYG@it=\textit\def\PYG@tc##1{\textcolor[rgb]{0.25,0.50,0.56}{##1}}}
\expandafter\def\csname PYG@tok@sh\endcsname{\def\PYG@tc##1{\textcolor[rgb]{0.25,0.44,0.63}{##1}}}
\expandafter\def\csname PYG@tok@kn\endcsname{\let\PYG@bf=\textbf\def\PYG@tc##1{\textcolor[rgb]{0.00,0.44,0.13}{##1}}}
\expandafter\def\csname PYG@tok@vi\endcsname{\def\PYG@tc##1{\textcolor[rgb]{0.73,0.38,0.84}{##1}}}
\expandafter\def\csname PYG@tok@ne\endcsname{\def\PYG@tc##1{\textcolor[rgb]{0.00,0.44,0.13}{##1}}}
\expandafter\def\csname PYG@tok@sb\endcsname{\def\PYG@tc##1{\textcolor[rgb]{0.25,0.44,0.63}{##1}}}
\expandafter\def\csname PYG@tok@bp\endcsname{\def\PYG@tc##1{\textcolor[rgb]{0.00,0.44,0.13}{##1}}}
\expandafter\def\csname PYG@tok@c\endcsname{\let\PYG@it=\textit\def\PYG@tc##1{\textcolor[rgb]{0.25,0.50,0.56}{##1}}}
\expandafter\def\csname PYG@tok@kc\endcsname{\let\PYG@bf=\textbf\def\PYG@tc##1{\textcolor[rgb]{0.00,0.44,0.13}{##1}}}
\expandafter\def\csname PYG@tok@kd\endcsname{\let\PYG@bf=\textbf\def\PYG@tc##1{\textcolor[rgb]{0.00,0.44,0.13}{##1}}}
\expandafter\def\csname PYG@tok@gs\endcsname{\let\PYG@bf=\textbf}
\expandafter\def\csname PYG@tok@nt\endcsname{\let\PYG@bf=\textbf\def\PYG@tc##1{\textcolor[rgb]{0.02,0.16,0.45}{##1}}}
\expandafter\def\csname PYG@tok@mf\endcsname{\def\PYG@tc##1{\textcolor[rgb]{0.13,0.50,0.31}{##1}}}
\expandafter\def\csname PYG@tok@mh\endcsname{\def\PYG@tc##1{\textcolor[rgb]{0.13,0.50,0.31}{##1}}}
\expandafter\def\csname PYG@tok@no\endcsname{\def\PYG@tc##1{\textcolor[rgb]{0.38,0.68,0.84}{##1}}}
\expandafter\def\csname PYG@tok@mo\endcsname{\def\PYG@tc##1{\textcolor[rgb]{0.13,0.50,0.31}{##1}}}

\def\PYGZbs{\char`\\}
\def\PYGZus{\char`\_}
\def\PYGZob{\char`\{}
\def\PYGZcb{\char`\}}
\def\PYGZca{\char`\^}
\def\PYGZam{\char`\&}
\def\PYGZlt{\char`\<}
\def\PYGZgt{\char`\>}
\def\PYGZsh{\char`\#}
\def\PYGZpc{\char`\%}
\def\PYGZdl{\char`\$}
\def\PYGZhy{\char`\-}
\def\PYGZsq{\char`\'}
\def\PYGZdq{\char`\"}
\def\PYGZti{\char`\~}
% for compatibility with earlier versions
\def\PYGZat{@}
\def\PYGZlb{[}
\def\PYGZrb{]}
\makeatother

\renewcommand\PYGZsq{\textquotesingle}

\begin{document}

\maketitle
\tableofcontents
\phantomsection\label{index::doc}


Contents:

Specifications of modules in project Tracemac.

Modules:


\chapter{Sources}
\label{index:sources}\label{index:welcome-to-tracemac-s-documentation}

\section{libs}
\label{modules:libs}\label{modules::doc}

\subsection{Switch\_Object module}
\label{Switch_Object:switch-object-module}\label{Switch_Object:module-Switch_Object}\label{Switch_Object::doc}\index{Switch\_Object (module)}
Created 2013-08-27

@author Mattias Måhl

Class Switch\_Object

This is an object to store Switchdata in.
\index{Sw\_Object (class in Switch\_Object)}

\begin{fulllineitems}
\phantomsection\label{Switch_Object:Switch_Object.Sw_Object}\pysiglinewithargsret{\strong{class }\code{Switch\_Object.}\bfcode{Sw\_Object}}{\emph{*ipaddress}}{}
Bases: \code{object}

Initilization of the Switch Object setting up the switch and
getting the information from the switch through SNMP.

Defining the tuples and variables to store the switch data.
\index{append\_mac\_to\_interface() (Switch\_Object.Sw\_Object method)}

\begin{fulllineitems}
\phantomsection\label{Switch_Object:Switch_Object.Sw_Object.append_mac_to_interface}\pysiglinewithargsret{\bfcode{append\_mac\_to\_interface}}{\emph{interface}, \emph{macaddress}}{}
function append\_mac\_to\_interface

a function to add mac-address to a specific interface. if interface not found add new interface and add the mac\_address to it.

\end{fulllineitems}

\index{append\_neighbor() (Switch\_Object.Sw\_Object method)}

\begin{fulllineitems}
\phantomsection\label{Switch_Object:Switch_Object.Sw_Object.append_neighbor}\pysiglinewithargsret{\bfcode{append\_neighbor}}{\emph{interface}, \emph{*args}}{}
function append\_neighbor

function to append Neighbor to the switchobjects array och neighbors.

\end{fulllineitems}

\index{check\_if\_alive() (Switch\_Object.Sw\_Object method)}

\begin{fulllineitems}
\phantomsection\label{Switch_Object:Switch_Object.Sw_Object.check_if_alive}\pysiglinewithargsret{\bfcode{check\_if\_alive}}{\emph{ipaddress}}{}
function check\_if\_alive

function to ping host to see if it's alive.

raises error if it's a fail.

\end{fulllineitems}

\index{check\_ip\_address() (Switch\_Object.Sw\_Object method)}

\begin{fulllineitems}
\phantomsection\label{Switch_Object:Switch_Object.Sw_Object.check_ip_address}\pysiglinewithargsret{\bfcode{check\_ip\_address}}{\emph{ipaddress}}{}
function check\_ip\_address

function to check if the ip-address provided is acceptable.

i.e. number(dot)number(dot)number(dot)number

return False if not and returns the same string if True

\end{fulllineitems}

\index{Sw\_Object.cl\_switch\_interface (class in Switch\_Object)}

\begin{fulllineitems}
\phantomsection\label{Switch_Object:Switch_Object.Sw_Object.cl_switch_interface}\pysiglinewithargsret{\strong{class }\bfcode{cl\_switch\_interface}}{\emph{interface}, \emph{mac\_address}}{}
Bases: \code{object}

class cl\_switch\_interface

Switch\_objects interface object to store mac-adresses assosiated with the interface.

\end{fulllineitems}

\index{Sw\_Object.cl\_switch\_neighbors (class in Switch\_Object)}

\begin{fulllineitems}
\phantomsection\label{Switch_Object:Switch_Object.Sw_Object.cl_switch_neighbors}\pysiglinewithargsret{\strong{class }\code{Sw\_Object.}\bfcode{cl\_switch\_neighbors}}{\emph{interface}}{}
Bases: \code{object}

class cl\_switch\_neighbores

Switch\_Objects neighbors object to store the switches neighbors and witch interface their on.
\index{interface (Switch\_Object.Sw\_Object.cl\_switch\_neighbors attribute)}

\begin{fulllineitems}
\phantomsection\label{Switch_Object:Switch_Object.Sw_Object.cl_switch_neighbors.interface}\pysigline{\bfcode{interface}\strong{ = 0}}
\end{fulllineitems}

\index{ip\_address (Switch\_Object.Sw\_Object.cl\_switch\_neighbors attribute)}

\begin{fulllineitems}
\phantomsection\label{Switch_Object:Switch_Object.Sw_Object.cl_switch_neighbors.ip_address}\pysigline{\bfcode{ip\_address}\strong{ = `'}}
\end{fulllineitems}

\index{name (Switch\_Object.Sw\_Object.cl\_switch\_neighbors attribute)}

\begin{fulllineitems}
\phantomsection\label{Switch_Object:Switch_Object.Sw_Object.cl_switch_neighbors.name}\pysigline{\bfcode{name}\strong{ = `'}}
\end{fulllineitems}

\index{remote\_interface (Switch\_Object.Sw\_Object.cl\_switch\_neighbors attribute)}

\begin{fulllineitems}
\phantomsection\label{Switch_Object:Switch_Object.Sw_Object.cl_switch_neighbors.remote_interface}\pysigline{\bfcode{remote\_interface}\strong{ = 0}}
\end{fulllineitems}


\end{fulllineitems}

\index{find\_my\_mac\_address() (Switch\_Object.Sw\_Object method)}

\begin{fulllineitems}
\phantomsection\label{Switch_Object:Switch_Object.Sw_Object.find_my_mac_address}\pysiglinewithargsret{\code{Sw\_Object.}\bfcode{find\_my\_mac\_address}}{\emph{mac\_address}}{}
function to search the Mac-table of the switch to find a specific MAC-adress.

\end{fulllineitems}

\index{find\_switch\_mac\_address() (Switch\_Object.Sw\_Object method)}

\begin{fulllineitems}
\phantomsection\label{Switch_Object:Switch_Object.Sw_Object.find_switch_mac_address}\pysiglinewithargsret{\code{Sw\_Object.}\bfcode{find\_switch\_mac\_address}}{}{}
function find\_switch\_mac\_address()

Do snmp request för oid: 1.3.6.1.4.1.11.2.14.11.5.1.1.6.0 (BaseMacAddress)

\end{fulllineitems}

\index{get\_interface() (Switch\_Object.Sw\_Object method)}

\begin{fulllineitems}
\phantomsection\label{Switch_Object:Switch_Object.Sw_Object.get_interface}\pysiglinewithargsret{\code{Sw\_Object.}\bfcode{get\_interface}}{\emph{interface}}{}
function get\_interface

searches registered interfaces and returns the interface\_object

\end{fulllineitems}

\index{get\_mac\_address\_list() (Switch\_Object.Sw\_Object method)}

\begin{fulllineitems}
\phantomsection\label{Switch_Object:Switch_Object.Sw_Object.get_mac_address_list}\pysiglinewithargsret{\code{Sw\_Object.}\bfcode{get\_mac\_address\_list}}{}{}
function get\_mac\_address\_list

Does a SNMP-request to gather the mac-address-table from this switch-object.

\end{fulllineitems}

\index{get\_neighbor\_at\_interface() (Switch\_Object.Sw\_Object method)}

\begin{fulllineitems}
\phantomsection\label{Switch_Object:Switch_Object.Sw_Object.get_neighbor_at_interface}\pysiglinewithargsret{\code{Sw\_Object.}\bfcode{get\_neighbor\_at\_interface}}{\emph{interface}}{}
function get\_neighbor\_at\_interface

function to get the neighbor at a specific interface.

searches registered neighbors and returns a hit.

\end{fulllineitems}

\index{get\_neighbors() (Switch\_Object.Sw\_Object method)}

\begin{fulllineitems}
\phantomsection\label{Switch_Object:Switch_Object.Sw_Object.get_neighbors}\pysiglinewithargsret{\code{Sw\_Object.}\bfcode{get\_neighbors}}{}{}
function get\_neighbores

gets the list of neighbors and stores them in the list `switch\_neighbors'

\end{fulllineitems}

\index{get\_switch\_data() (Switch\_Object.Sw\_Object method)}

\begin{fulllineitems}
\phantomsection\label{Switch_Object:Switch_Object.Sw_Object.get_switch_data}\pysiglinewithargsret{\code{Sw\_Object.}\bfcode{get\_switch\_data}}{}{}
Do snmp-recuest to get the system name of the target switch.

\end{fulllineitems}

\index{is\_interface\_a\_neighbor() (Switch\_Object.Sw\_Object method)}

\begin{fulllineitems}
\phantomsection\label{Switch_Object:Switch_Object.Sw_Object.is_interface_a_neighbor}\pysiglinewithargsret{\code{Sw\_Object.}\bfcode{is\_interface\_a\_neighbor}}{\emph{interface}}{}
function is\_interface\_a\_neighbor

Check to see if the interface has neighbors registered.

Returns True or False

\end{fulllineitems}


\end{fulllineitems}

\index{sw\_error}

\begin{fulllineitems}
\phantomsection\label{Switch_Object:Switch_Object.sw_error}\pysiglinewithargsret{\strong{exception }\code{Switch\_Object.}\bfcode{sw\_error}}{\emph{error\_msg}}{}
Bases: \code{Exception}

\end{fulllineitems}



\subsection{Trace\_Functions module}
\label{Trace_Functions:trace-functions-module}\label{Trace_Functions:module-Trace_Functions}\label{Trace_Functions::doc}\index{Trace\_Functions (module)}
Created 2013-08-27

@author Mattias Måhl

Class Tracefunctions

Functions to administrate search for MAC-adress
\index{Tracefunctions (class in Trace\_Functions)}

\begin{fulllineitems}
\phantomsection\label{Trace_Functions:Trace_Functions.Tracefunctions}\pysigline{\strong{class }\code{Trace\_Functions.}\bfcode{Tracefunctions}}
Bases: \code{object}
\index{Tracefunctions.Trace\_arguments (class in Trace\_Functions)}

\begin{fulllineitems}
\phantomsection\label{Trace_Functions:Trace_Functions.Tracefunctions.Trace_arguments}\pysigline{\strong{class }\bfcode{Trace\_arguments}}
Bases: \code{object}

Class Trace\_arguments
Object to store the supplied arguments.
\index{dump\_file (Trace\_Functions.Tracefunctions.Trace\_arguments attribute)}

\begin{fulllineitems}
\phantomsection\label{Trace_Functions:Trace_Functions.Tracefunctions.Trace_arguments.dump_file}\pysigline{\bfcode{dump\_file}\strong{ = `'}}
\end{fulllineitems}

\index{in\_file (Trace\_Functions.Tracefunctions.Trace\_arguments attribute)}

\begin{fulllineitems}
\phantomsection\label{Trace_Functions:Trace_Functions.Tracefunctions.Trace_arguments.in_file}\pysigline{\bfcode{in\_file}\strong{ = `'}}
\end{fulllineitems}

\index{start\_ip\_address (Trace\_Functions.Tracefunctions.Trace\_arguments attribute)}

\begin{fulllineitems}
\phantomsection\label{Trace_Functions:Trace_Functions.Tracefunctions.Trace_arguments.start_ip_address}\pysigline{\bfcode{start\_ip\_address}\strong{ = `'}}
\end{fulllineitems}

\index{target\_ip\_address (Trace\_Functions.Tracefunctions.Trace\_arguments attribute)}

\begin{fulllineitems}
\phantomsection\label{Trace_Functions:Trace_Functions.Tracefunctions.Trace_arguments.target_ip_address}\pysigline{\bfcode{target\_ip\_address}\strong{ = `'}}
\end{fulllineitems}

\index{target\_mac\_address (Trace\_Functions.Tracefunctions.Trace\_arguments attribute)}

\begin{fulllineitems}
\phantomsection\label{Trace_Functions:Trace_Functions.Tracefunctions.Trace_arguments.target_mac_address}\pysigline{\bfcode{target\_mac\_address}\strong{ = `'}}
\end{fulllineitems}

\index{verbose (Trace\_Functions.Tracefunctions.Trace\_arguments attribute)}

\begin{fulllineitems}
\phantomsection\label{Trace_Functions:Trace_Functions.Tracefunctions.Trace_arguments.verbose}\pysigline{\bfcode{verbose}\strong{ = False}}
\end{fulllineitems}


\end{fulllineitems}

\index{Tracefunctions.Trace\_result (class in Trace\_Functions)}

\begin{fulllineitems}
\phantomsection\label{Trace_Functions:Trace_Functions.Tracefunctions.Trace_result}\pysigline{\strong{class }\code{Tracefunctions.}\bfcode{Trace\_result}}
Bases: \code{object}

Class Trace\_result
Object to store a single result from the search.
This stores the path the program took to find the port witch has the mac-adress.
\index{SW\_O (Trace\_Functions.Tracefunctions.Trace\_result attribute)}

\begin{fulllineitems}
\phantomsection\label{Trace_Functions:Trace_Functions.Tracefunctions.Trace_result.SW_O}\pysigline{\bfcode{SW\_O}\strong{ = {[}{]}}}
\end{fulllineitems}

\index{failed (Trace\_Functions.Tracefunctions.Trace\_result attribute)}

\begin{fulllineitems}
\phantomsection\label{Trace_Functions:Trace_Functions.Tracefunctions.Trace_result.failed}\pysigline{\bfcode{failed}\strong{ = False}}
\end{fulllineitems}

\index{search\_ip (Trace\_Functions.Tracefunctions.Trace\_result attribute)}

\begin{fulllineitems}
\phantomsection\label{Trace_Functions:Trace_Functions.Tracefunctions.Trace_result.search_ip}\pysigline{\bfcode{search\_ip}\strong{ = `'}}
\end{fulllineitems}

\index{search\_mac (Trace\_Functions.Tracefunctions.Trace\_result attribute)}

\begin{fulllineitems}
\phantomsection\label{Trace_Functions:Trace_Functions.Tracefunctions.Trace_result.search_mac}\pysigline{\bfcode{search\_mac}\strong{ = `'}}
\end{fulllineitems}

\index{trace\_end (Trace\_Functions.Tracefunctions.Trace\_result attribute)}

\begin{fulllineitems}
\phantomsection\label{Trace_Functions:Trace_Functions.Tracefunctions.Trace_result.trace_end}\pysigline{\bfcode{trace\_end}\strong{ = `'}}
\end{fulllineitems}


\end{fulllineitems}

\index{chk\_system\_args() (Trace\_Functions.Tracefunctions method)}

\begin{fulllineitems}
\phantomsection\label{Trace_Functions:Trace_Functions.Tracefunctions.chk_system_args}\pysiglinewithargsret{\code{Tracefunctions.}\bfcode{chk\_system\_args}}{\emph{argv}}{}
function chk\_system\_args
function to check if argumest supplied are correct and that mandatory argurments are supplied.

\end{fulllineitems}

\index{fix\_macaddress() (Trace\_Functions.Tracefunctions method)}

\begin{fulllineitems}
\phantomsection\label{Trace_Functions:Trace_Functions.Tracefunctions.fix_macaddress}\pysiglinewithargsret{\code{Tracefunctions.}\bfcode{fix\_macaddress}}{\emph{MAC}}{}
function fix\_macaddres
this function fixes the mac-address to be exactly the same even if the user supplies it in different formats.
i.e 121212-121212 will become 121212121212 and likewise 12:12:12:12:12:12 will become 121212121212.

\end{fulllineitems}

\index{get\_mac\_address\_from\_ip() (Trace\_Functions.Tracefunctions method)}

\begin{fulllineitems}
\phantomsection\label{Trace_Functions:Trace_Functions.Tracefunctions.get_mac_address_from_ip}\pysiglinewithargsret{\code{Tracefunctions.}\bfcode{get\_mac\_address\_from\_ip}}{\emph{ipaddress}}{}
function get\_mac\_address\_from\_ip
function to arping an ip address to get the Mac-address assosiated woth it.
It's important that the target ip address is on the same network as the machine running the program and NOT routed!
If it's routed the routers mac address will be the one reported by the arping!

\end{fulllineitems}

\index{get\_system\_args() (Trace\_Functions.Tracefunctions method)}

\begin{fulllineitems}
\phantomsection\label{Trace_Functions:Trace_Functions.Tracefunctions.get_system_args}\pysiglinewithargsret{\code{Tracefunctions.}\bfcode{get\_system\_args}}{\emph{argv}}{}
function get\_system\_args
function to parse system arguments in cli-envioronment.
Args:
\begin{quote}

-h, --help=           Display helptext for cli-command.
\begin{description}
\item[{-i, --ipaddress=      Target ip-address to find in the network.}] \leavevmode
\emph{note} this implies access to both mgmnt- and target network (if separate)

\end{description}

-s, --startingip=     IP-address of the first switch in the cascade.

-o, --out=            Output logging top specified file.

-m, --macaddress=     Target MAC-address to find in the network.

-v, --verbose=        Enebles verbose output to standard output and logging file if `-o/-out= is used.
\begin{optionlist}{3cm}
\item [-f, -{-}in-file]  
Enables function to loop through a list of targets in a text file.
\emph{note} One target per line.
\end{optionlist}
\end{quote}

\end{fulllineitems}

\index{ping\_my\_address() (Trace\_Functions.Tracefunctions method)}

\begin{fulllineitems}
\phantomsection\label{Trace_Functions:Trace_Functions.Tracefunctions.ping_my_address}\pysiglinewithargsret{\code{Tracefunctions.}\bfcode{ping\_my\_address}}{\emph{ipaddress}, \emph{cnt}}{}
function ping\_my\_address
simple function to ping the target IP-address to keep the Mac-address table up-to-date.

\end{fulllineitems}

\index{printhelp() (Trace\_Functions.Tracefunctions method)}

\begin{fulllineitems}
\phantomsection\label{Trace_Functions:Trace_Functions.Tracefunctions.printhelp}\pysiglinewithargsret{\code{Tracefunctions.}\bfcode{printhelp}}{}{}
function printhelp
duh!

\end{fulllineitems}


\end{fulllineitems}



\subsection{frm\_main module}
\label{frm_main:module-frm_main}\label{frm_main:frm-main-module}\label{frm_main::doc}\index{frm\_main (module)}
Created 2016-04-12
@author: Mattias Måhl
\index{frm\_main (class in frm\_main)}

\begin{fulllineitems}
\phantomsection\label{frm_main:frm_main.frm_main}\pysigline{\strong{class }\code{frm\_main.}\bfcode{frm\_main}}
Bases: \code{tkinter.Tk}

Start object to render the application window.

use:

import libs.frm\_main as mainwindow

if \_\_name\_\_ == ``\_\_main\_\_'':
\begin{quote}

app = mainwindow.App()

app.mainloop()
\end{quote}
\index{createWidgets() (frm\_main.frm\_main method)}

\begin{fulllineitems}
\phantomsection\label{frm_main:frm_main.frm_main.createWidgets}\pysiglinewithargsret{\bfcode{createWidgets}}{\emph{frame}}{}
Creates and lays out the widgets for the mainwindow.

\end{fulllineitems}

\index{quit() (frm\_main.frm\_main method)}

\begin{fulllineitems}
\phantomsection\label{frm_main:frm_main.frm_main.quit}\pysiglinewithargsret{\bfcode{quit}}{\emph{*event}}{}
\end{fulllineitems}


\end{fulllineitems}



\subsection{tracemac module}
\label{tracemac:module-tracemac}\label{tracemac::doc}\label{tracemac:tracemac-module}\index{tracemac (module)}
** TraceMac
** Traces a mac-address to a specific port in a switch.
** Version: 1.7.2
** License: GPLv2
\index{pr() (in module tracemac)}

\begin{fulllineitems}
\phantomsection\label{tracemac:tracemac.pr}\pysiglinewithargsret{\code{tracemac.}\bfcode{pr}}{\emph{pr\_str}}{}
function pr
function to print out messages to stdout using a specified format.
This is used for verbose output to stdout and logfile.

\end{fulllineitems}

\index{printout() (in module tracemac)}

\begin{fulllineitems}
\phantomsection\label{tracemac:tracemac.printout}\pysiglinewithargsret{\code{tracemac.}\bfcode{printout}}{\emph{msg}, \emph{*w}}{}
function printout
stardart output to stdout and logfile.

\end{fulllineitems}

\index{read\_infile() (in module tracemac)}

\begin{fulllineitems}
\phantomsection\label{tracemac:tracemac.read_infile}\pysiglinewithargsret{\code{tracemac.}\bfcode{read\_infile}}{\emph{filename}}{}
function read\_infile
this function serves to read the input file an store the targets for the search engine.

\end{fulllineitems}

\index{split\_string\_into\_chunks() (in module tracemac)}

\begin{fulllineitems}
\phantomsection\label{tracemac:tracemac.split_string_into_chunks}\pysiglinewithargsret{\code{tracemac.}\bfcode{split\_string\_into\_chunks}}{\emph{text}, \emph{length=94}}{}
function split\_string\_into\_chunks
this function serves to sprit output into chunks that's under specified length.
Default value is 94 chars.

\end{fulllineitems}

\index{vreport() (in module tracemac)}

\begin{fulllineitems}
\phantomsection\label{tracemac:tracemac.vreport}\pysiglinewithargsret{\code{tracemac.}\bfcode{vreport}}{\emph{header}, \emph{*msg}}{}
function vreport
creates and outputs verbose output of progress.

\end{fulllineitems}

\index{write\_to\_file() (in module tracemac)}

\begin{fulllineitems}
\phantomsection\label{tracemac:tracemac.write_to_file}\pysiglinewithargsret{\code{tracemac.}\bfcode{write\_to\_file}}{\emph{line}}{}
function write\_to\_file
if there is a file specified in options this will write to it.

\end{fulllineitems}



\chapter{Indices and tables}
\label{index:indices-and-tables}\begin{itemize}
\item {} 
\DUrole{xref,std,std-ref}{genindex}

\item {} 
\DUrole{xref,std,std-ref}{modindex}

\item {} 
\DUrole{xref,std,std-ref}{search}

\end{itemize}


\renewcommand{\indexname}{Python Module Index}
\begin{theindex}
\def\bigletter#1{{\Large\sffamily#1}\nopagebreak\vspace{1mm}}
\bigletter{f}
\item {\texttt{frm\_main}}, \pageref{frm_main:module-frm_main}
\indexspace
\bigletter{s}
\item {\texttt{Switch\_Object}}, \pageref{Switch_Object:module-Switch_Object}
\indexspace
\bigletter{t}
\item {\texttt{Trace\_Functions}}, \pageref{Trace_Functions:module-Trace_Functions}
\item {\texttt{tracemac}}, \pageref{tracemac:module-tracemac}
\end{theindex}

\renewcommand{\indexname}{Index}
\printindex
\end{document}
